\section{问题重述}
\subsection{问题背景}
在中国古代,丝绸之路作为中西方文化交流的主要渠道,玻璃作为当时最具代表性的身份象征,体现其拥有者的尊贵。早期玻璃是从西亚和埃及地区常被制作成珠形饰品传入我国。我国古代大能通过研究玻璃的成分以及其原理在本土就地取材制作,虽然外观相似,但是化学成分却不相同。

玻璃的主要原料是石英砂,主要化学成分是二氧化硅(SiO2)。在煅烧玻璃时,需要添加一些辅助材料作为助熔剂,因此会产生很多其他的化学成分。石灰石煅烧之后会转化为氧化钙(CaO)。铅钡玻璃在烧制的过程中加入铅矿石作为助熔剂,其中氧化铅(PbO)、氧化钡(BaO)的含量较高,通常被认为是我们自己发明的玻璃品种,楚文化的玻璃就是以铅钡玻璃为主。其中,高钾玻璃是以含钾量高的物质如草木灰作为助熔剂烧制而成的。

\subsection{问题的提出}
由于古代的保存手段比较低端,因此古代玻璃极易受埋藏环境的影响而风化。在风化过程中,内部元素与环境元素进行大量交换,导致其成分比例发生变化,从而影响对其类别的正确判断。因此,需要通过数字数据来对玻璃的风化状态以及化学成分进行分析。

\subsection{问题的分解}
\subsubsection{问题一分解}
\begin{enumerate}
	\item 分析玻璃文物的表面风化与玻璃类型、纹饰和颜色之间的关系;
	\item 分析文物样品表面有无风化化学成分含量的统计规律;
	\item 预测风化点检测数据在风化前的化学成分含量;
\end{enumerate}

\subsubsection{问题二分解}
\begin{enumerate}
	\item 分析高钾玻璃、铅钡玻璃的分类规律;
	\item 对每个玻璃类别进行亚分类,给出划分方法和结果;
	\item 分析分类结果的合理性和敏感性;
\end{enumerate}

\subsubsection{问题三分解}
\begin{enumerate}
	\item 对附件表单3中的未知类别玻璃文物进行鉴别;
	\item 对分类结果敏感性进行分析;
\end{enumerate}

\subsubsection{问题四分解}
\begin{enumerate}
	\item 分析不同类别玻璃文物中化学成分之间的关系;
	\item 比较不同类别玻璃文物中化学成分的差异性;
\end{enumerate}