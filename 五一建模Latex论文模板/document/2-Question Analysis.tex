\section{问题分析}

\subsection{数据集分析}
样本数据的采集方式有很多种,不同的方式会得到不同质量的样本数据,一般情况下我们获取的原始数据并不能直接为我们所用,需要根据我们所分析的问题通过预处理的方法将数据转变成合适的干净数据。数据预处理一般包含数据清洗、数据的缺失值和异常值处理、特征工程等几个步骤。本文中涉及到的数据处理、分析以及模型的建立与评估均采用Python3.7版本来实现,其中主要使用Python中的Pandas、Numpy、Seaborn、Matplotlib、Sklearn等库\textsuperscript{\cite{b6}}。

本文主要通过SPSSPRO软件中的Notebook应用来建立jupyter文件,然后读取给定的数据集,导入建模所需的工具库。

\subsection{缺失值处理}
对于缺失值的处理有三种方法,分别是缺失值插补法、删除样本法与直接使用缺失值的特征。插补法是根据样本数据情况使用均值、众数或者相邻样本来补充缺失值;删除样本法是对于样本数据的变量缺失过多或剩余变量无明显特征,无法体现出样本的特征信息,删除该样本对整体结构无法造成影响;直接使用缺失值特征是在某些特殊情况下,将缺失值映射为一个类别特征,则可不对其进行处理。

通过isnull()函数查看数据集是否包含缺失值以及异常值,结果如表\ref{t1}。
\begin{table}[!htbp]
	\centering
	\setlength{\abovecaptionskip}{3pt}%caption与表格之间的距离
	\caption{缺失值检测}
	\vspace{1pt}
	\label{t1}
	\resizebox{\textwidth}{!}{
		\begin{tabular}{cc}
			\toprule[1.5pt]
			\makebox[0.4\textwidth]{特证名} & \makebox[0.5\textwidth]{空值个数} \\
			\midrule
			文物编号 & 0 \\
			纹饰 & 0 \\
			类型 & 0 \\
			颜色 & 4 \\
			表面风化 & 0 \\
			
			\bottomrule[1.5pt]
		\end{tabular}
	}
\end{table}

\subsubsection{问题一分析}
第一小问需要进行差异性分析:由于所有特征变量为定类变量,因此进行卡方检验分析确定自变量与因变量之间的关系。通过SPSSPRO进行求解。分析显著性p值是否小于0.05来分析其差异性关系。

第二小问是变化规律分析:通过计算样本数据的均值、计数、中位数、标准差、最大值、最小值和求和共七个统计量进行描述性统计分析、散点图统计分析、正态分布检验直方图等可视化展现等,总结变化情况。

第三小问是预测化学成分,根据风化前后的数据规律,总结各个化学成分的变化情况,找出之间的关系并预测风化前的含量。

\subsubsection{问题二分析}
第一小问需要针对高钾玻璃和铅钡玻璃不同化学成分的数值进行统计,找到分类的依据。

第二小问需要在第一小问的基础上进行亚类划分,观察化学成分在风化前后的颜色变化、纹理变化等,并给出相关的分类依据。

第三小问需要对数据进行灵敏性检验,并给出合理性依据。

\subsubsection{问题三分析}
第一小问需要将附件表单三中附件中有无风化的情况进行分类讨论,结合问题2中模型的结论,对表单三中不同类型的玻璃进行分类。

第二小问通过对某一类化学成分进行增加或减少,观察分类情况是否发生变化,给出模型的稳定性结论。

\subsubsection{问题四分析}
第一小问需要选取具有代表性的化学成分进行灰色关联分析,与问题一第二小问的区别在于少了一个有无风化的条件,选择化学成分占比最大的作为因变量,其余作为自变量,建立灰色关联分析模型,计算其灰色关联度的情况。

第二小问需要对不同类别的玻璃进行显著性检验,观察两种玻璃之间的差异性。