\begin{thebibliography}{9}%宽度9
	\bibitem{b1}刘露诗. 基于机器学习与缺失值插补技术的海底硫化物成矿定量预测[D].吉林大学,2022.DOI:10.27162/d.cnki.gjlin.2022.000002.
	\bibitem{b2}卓炜杰.函数型聚类分析方法及其应用研究[D].浙江工商大学,2018.DOI:10.27462/d.cnki.ghzhc.2018.000045.
	\bibitem{b3}王凌妍,张鑫雨,许胜楠,王禹力,甄志龙.逻辑回归的敏感性分析及在特征选择中的应用[J].信息记录材料,2022,23(07):30-33.DOI:10.16009/j.cnki.cn13-1295/tq.2022.07.051.
	\bibitem{b4}于群,霍筱东,何剑,李琳,张建新,冯煜尧.基于斯皮尔曼相关系数和系统惯量的中国电网停电事故趋势预测[J/OL].中国电机工程学报:1-12[2022-09-18].http://kns.cnki.net/kcms/detail/11.2107.TM.20220824.1625.012.html
	\bibitem{b5}Collins M, Schapire R E, Singer Y.Logistic Regression, AdaBoost and Bregman Dist ances[J]. Machine Learning Journal,2002,48(1-3):253-285.
	\bibitem{b6}李航.统计学习方法[M].北京:清华大学出版社,2012:116-123.
	\bibitem{b7}Varshneya Arun K.,Macrelli Guglielmo,Yoshida Satoshi,Kim Seong H.,Ogrinc Andrew L.,Mauro John C.. Indentation and abrasion in glass products: Lessons learned and yet to be learned[J]. International Journal of Applied Glass Science,2022,13(3).
	\bibitem{b8}卢树强.数学建模的算法创新与实践应用——评《数学建模算法与应用(第3版)》[J].现代雷达,2022,44(03):111.
\end{thebibliography}