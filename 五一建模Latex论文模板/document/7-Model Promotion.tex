\section{模型评价与推广}
\subsection{Spearman相关性系数分析}
优点:
\begin{enumerate}
	\item 既可以使用线性相关系数又适用于非线性相关系数
	\item 在变量值没有变化的情况下,也不会出现像皮尔森系数那样分母为0而无法计算的情况。另外,即使出现异常值,由于异常值的秩次通常不会有明显的变化
	\item 可以测量两个定序数据
\end{enumerate}

\subsection{逻辑回归}
优点:
\begin{enumerate}
	\item 模型训练速度快,计算量仅仅只和特征的数目相关
	\item 模型易于理解,解释性好,从特征的权重可以看到不同的特征对最后结果的影响
	\item 适用于二分类问题,特征不需要缩放
	\item 内存占用资源小
\end{enumerate}

缺点:
\begin{enumerate}
	\item 数据不平衡时 不能得到处理
	\item 准确率较低,形式简单以至难以去拟合数据的真实分布
	\item 逻辑回归无法筛选其特征,gbdt后才可以逻辑回归
	\item 对多重共线性数据过为敏感		
\end{enumerate}